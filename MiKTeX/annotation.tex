\sectioncentered*{Аннотация}
\thispagestyle{empty}

\begin{center}
  \begin{minipage}{0.82\textwidth}
    на дипломный проект <<Приложение для автоматизации процесса и поддержкибезопасной сдачи в аренду товаров для физических лиц>> студента УО <<Белорусский государственный университет информатики и радиоэлектроники>> Дроздова~А.\,С.
  \end{minipage}
\end{center}

\emph{Ключевые слова}: электронная коммерция; программный модуль; вывод структуры сети по данным; программное средство; веб-приложение.

\vspace{4\parsep}


Целью дипломного проекта является разработка приложение для автоматизации процесса и поддержкибезопасной сдачи в аренду товаров для физических лиц.
Во введении производится ознакомление с проблемой, решаемой в дипломном проекте.

В первой главе производится обзор предметной области проблемы решаемой в данном дипломном проекте.
Приводятся необходимые теоретические сведения, а также производится обзор существующих аналогов.
ФОрмулируются требования к программному обеспечени.

Во второй главе производится краткий обзор технологий, использованных для реализации ПО в рамках дипломного проекта.

В третьей главе производится проектирование ПО.
Описываются его составные части и особенности.

В пятой главе производится описание разработкри ПО.
Описывается разработка серверной и клиентской части приоложения.
Описывается контейнеризация приложения в Docker.

В шествой главе производится технико"=экономическое обоснование разработки.

В заключении подводятся итоги и делаются выводы по дипломному проекту, а также описывается дальнейший план развития проекта.

Дипломный проект выполнен самостоятельно и проверен в системе «Антиплагиат».
Процент оригинальности соответствует норме, установленной кафедрой информатики.
Цитирования обозначены ссылками на публикации, указанные в «Списке литературы».

% \begin{figure}[h]
%   \centering
%   \includegraphics[scale=0.3]{antiplagiat.png}
% \end{figure}

\clearpage
