\sectioncentered*{Заключение}
\addcontentsline{toc}{section}{Заключение}

В настоящем документе описан цикл разработки программного средства позволяющего физическим лицам сдавать в аренду собственный вещи другим физическим лицам.

В ходе разработки дипломного проекта для обеспечения стабильности и расширяемости были использованы современные методы и технологии разработки, которые нашли
применение при реализации программного продукта: код программы был разбит на множество компонентов, каждый отвечающий за своё назначение, все технологии обновлены до последней версии, проект использует новейшие
возможности каждой из них, что при дальнейшей поддержке приложения сильно упрощает разработку.
Такой подход обеспечивает дополнение и изменение кода без влияния на другие части проекта, а также возможность использовать полученный код в будущих проектах.
Был произведен анализ предметной области, анализ аналогов, разобрана сфера электронной коммерции и реализован полнофункциональный веб"=сайт, полностью готовый к применению.
Разработанный сайт ориентирован на аудиторию, заинтересованную в покупке и сдачи собственных вещей в аренду.

Была решена главная задача данного проекта, которая рознит реализованное приложение от обычного интернет-магазина, а именно процесс покупки товара в аренду.
На данном этапе было решено организовать процесс покупки таким образом, что один заказ может состоять из нескольких вещей, однако у всех этих вещей одинаковые сроки сдачи.


Основной целью разработки программы была реализация возможности физическим лицам сдавать собственные вещи в аренду.
В ходе работы над дипломным проектом эта цель была успешно достигнута.

В дальнейшем времени для развития сайта возможна доработка его интерфейса и функций, что повысит информативность, доступность, привлекательность и удобство пользования.
Так же, возможна реализация отдельного мобильного приложения, которая потребует только реализации клиентского приложения.
С точки зрения серверной части приложений возможна добавление реализации выбора срока выдачи товара в аренду.
Так как на данном этапе приложение является прототипом и разработано на монолитной архитектуре, в будущем возможен переход на микросервисную архитектуру.
