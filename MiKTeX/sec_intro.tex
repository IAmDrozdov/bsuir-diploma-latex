\sectioncentered*{Введение}
\addcontentsline{toc}{section}{Введение}
\label{sec:intro}

Четвертая промышленная революция знаменуется тем, что любые произведенные предметы, будь то чайник или станок, становятся Smart Connected Products — «умными» и подключенными продуктами.
Это открывает новые возможности для производителей: дистанционное обслуживание, ремонт только при необходимости, сбор и анализ данныхво время эксплуатации для совершенствования следующих модификаций.
Самое главное изменение касается трансформации бизнес-модели. Если раньше поставщики оборудования мыслили объемами поставок плюс доходами от обслуживания, то теперь самые прогрессивные из них переходят на сервисную бизнес-модель.
В этой модели клиенты платят только за результат работы и фактическое использование конкретного станка или машины.
Это подобно тому, как мы платим за тепло в квартире зимой, а не покупаем трубы и горячую воду в них[1].
Эти расходы мы несем только по факту получения услуги: летом затраты на отопление равны нулю.
