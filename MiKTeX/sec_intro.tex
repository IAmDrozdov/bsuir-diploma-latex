\sectioncentered*{Введение}
\addcontentsline{toc}{section}{Введение}
\label{sec:intro}

Четвертая промышленная революция знаменуется тем, что любые произведенные предметы, будь то чайник или станок, становятся Smart Connected Products — «умными» и подключенными продуктами.
Самое главное изменение касается трансформации бизнес-модели. Если раньше поставщики оборудования мыслили объемами поставок плюс доходами от обслуживания, то теперь самые прогрессивные из них переходят на сервисную бизнес-модель.
В этой модели клиенты платят только за результат работы и фактическое использование конкретного станка или машины.
Это подобно тому, как мы платим за тепло в квартире зимой, а не покупаем трубы и горячую воду в них[1].
Эти расходы мы несем только по факту получения услуги: летом затраты на отопление равны нулю.

Актуальность выбранной темы обсуловлена большим количеством сервисов по аренде вещей на рынке, разделенными по определенным сегментам.
Однако данные сервисы в своей системе координат не предусматриваю физическое лицо как продавца, а только лишь клиента.

Проанализировав рынок электронной комерции в Республике Беларусь было принято решение о разработке приложения для организации продажи вещей в аренду между физическими лицами.

Целью данного дипломного проекта является разработка приложения, позволяющего физическим лицам сдавать в аренду собственный вещи другим физическим лицам.
Данное приложение представляет из себя 2 модуля: серверный - обработка данных и их хранений и клиентский - визуальный интерфейс к серверной части.
Приложение будет работать в рамках определенного предприятия и выполнять функции основной площадки для ведения коммерческой деятельности.

Задачи, решаемые в данном проекте:
\begin{itemize}
    \item Провести анализ аналогов;
    \item Провести обзор web-инструментов, которые предоставляют для разработки веб"=приложений;
    \item Спроектировать и разработать приложение для автоматизации процесса и поддержки безопасной сдачи в аренду товаров для физических лиц.
\end{itemize}
