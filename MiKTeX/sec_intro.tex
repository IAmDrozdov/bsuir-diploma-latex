\sectioncentered*{Введение}
\addcontentsline{toc}{section}{Введение}
\label{sec:intro}

Четвертая промышленная революция знаменуется тем, что любые произведенные предметы, будь то чайник или станок, становятся Smart Connected Products — «умными» и подключенными продуктами.
Это открывает новые возможности для производителей: дистанционное обслуживание, ремонт только при необходимости, сбор и анализ данныхво время эксплуатации для совершенствования следующих модификаций.
Самое главное изменение касается трансформации бизнес-модели. Если раньше поставщики оборудования мыслили объемами поставок плюс доходами от обслуживания, то теперь самые прогрессивные из них переходят на сервисную бизнес-модель.
В этой модели клиенты платят только за результат работы и фактическое использование конкретного станка или машины.
Это подобно тому, как мы платим за тепло в квартире зимой, а не покупаем трубы и горячую воду в них.
Эти расходы мы несем только по факту получения услуги: летом затраты на отопление равны нулю.

Если свести все преимущества, которые получает заказчик от сервисной модели, в один список, получится следующее.

\begin{itemize}
    \item Возможность оплаты только за фактически потребленные качественные услуги поставщика.
    Таким образом, заказчик платит только за то, что ему нужно, и в тех объёмах, которые он использовал.
    \item Отсутствие непредсказуемых затрат во время эксплуатации, в том числе снижение риска получить некачественное оборудование.
    Так как теперь обслуживание оборудования — это забота производителя, заказчик может не беспокоиться о затратах на внезапный ремонт.
    \item Перевод затрат на дорогую технику из категории капитальных в операционные — то есть заказчику не требуется занимать средства или выводить деньги из оборота.
    Таким образом, общие затраты на эксплуатацию оборудования снижаются.
    \item Также можно рассчитывать на то, что коэффициент технической готовности к работе у такого оборудования выше.
    Ведь для производителя выгодно, чтобы его продукция работала как можно дольше и не ломалась, чтобы затраты на ее обслуживание были минимальными.
    Эти требования закладываются еще в момент проектирования.
    Тогда как в классической схеме, когда производитель просто продает физические объекты, в его интересах обратное.
    Идеальная ситуация для него — это когда машина ломается сразу же после истечения гарантийного срока эксплуатации.
    \end{itemize}

    Схема поставок по сервисной модели кажется на первый взгляд выгодной только потребителям.
    На самом деле для производителя это также возможность получать больше прибыли.

    Во-первых, за счет повышения маржинальности поставляемых операций, в том числе на премиальных выплатах за достижение результата.
    К примеру, команда «Формулы-1», использующая двигатели Rolls-Royce в своих болидах, может выплатить поставщику дополнительное вознаграждение в случае победы на гонках.

    Во-вторых, за счет сокращения затрат на обслуживание — благодаря дистанционному мониторингу состояния оборудования — можно отправлять ремонтные бригады только тогда, когда что-то действительно сломалось или скоро выйдет из строя.
    Кстати, предиктивная аналитика помогает также избежать более серьезных поломок, если дорогой прибор отремонтировать еще до того, как он дал сбой.
