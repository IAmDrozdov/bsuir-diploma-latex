\sectioncentered*{Перечень условных обозначений, символов и терминов}
\addcontentsline{toc}{section}{Перечень условных обозначений, символов и терминов}

В настоящей пояснительной записке применяются следующие определения и сокращения.

ORM – технология программирования, которая связывает базы данных с концепциями объектно-ориентированных языков программирования.

MVC – Model-View-Controller – модель-представление-контроллер.

REST – Representation State Transfer – передача состояния представления.

API – Application Programming Interface – интерфейс программирования приложений.

OTP - One-time password - это пароль, действительный только для одного сеанса аутентификации.

CSRF - Cross-Site Request Forgery - это межсайтовая подделка запроса, вид атак на посетителей веб-сайтов, использующий недостатки протокола HTTP.

SQL-инъекции - один из распространённых способов взлома сайтов и программ, работающих с базами данных, основанный на внедрении в запрос произвольного SQL-кода.

XSS - Cross-site scripting - тип атаки на веб-системы, заключающийся во внедрении в выдаваемую веб-системой страницу вредоносного кода.

Кликджекинг - обманная технология, основанная на размещении вызывающих какие-то действия невидимых элементов на сайте поверх видимых активных.

DRY - Don't repeat yourself - это принцип разработки программного обеспечения, нацеленный на снижение повторения информации различного рода.

E2E-тесты - End-to-end - полноценное тестирование системы, эмолируя пользовательскую среду.

WAR - формат файла, описывающий, как полное веб-приложение упаковывается.

CI / CD - continuous integration / continuous delivery - Концепция непрерывной интеграции и доставки.

DOM - Hypertext Markup Language - это независящий от платформы и языка программный интерфейс, позволяющий программам и скриптам получить доступ к содержимому HTML.
