\sectioncentered*{ПЕРЕЧЕНЬ УСЛОВНЫХ ОБОЗНАЧЕНИЙ, СИМВОЛОВ И ТЕРМИНОВ}
\thispagestyle{empty}

В настоящей пояснительной записке применяются следующие опреде- ления и сокращения.

ПО - программное обеспечение.

ORM – технология программирования, которая связывает базы данных с концепциями объектно-ориентированных языков программирования.

URL – Uniform Resource Locator – единый указатель ресурсов.

ООП – объектно-ориентированное программирование.

SQL – structured query language – язык структурированных запросов.

HTTP – Hypertext Transfer Protocol – протокол передачи гипертекста.

Паттерн – эффективный способ решения характерных задач проектирования, в частности проектирования компьютерных программ.

MVC – Model-View-Controller – модель-представление-контроллер.

БД – база данных.

СУБД – система управления базами данных.

REST – Representation State Transfer – передача состояния представления.

API – Application Programming Interface – интерфейс программирования приложений.

JSON – JavaScript Object Notation – текстовый формат обмена данными, основанный на языке JavaScript.

UI – User Interface – графический интерфейс пользователя.

OTP - это пароль, действительный только для одного сеанса аутентификации.
