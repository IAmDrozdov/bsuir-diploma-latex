% % Зачем: Изменение надписи для списка литературы
% % Почему: Пункт 2.8.1 Требований по оформлению пояснительной записки.
% \renewcommand{\bibsection}{\sectioncentered*{Cписок использованных источников}}
% % \phantomsection\pagebreak% исправляет нумерацию в документе и исправляет гиперссылки в pdf
% \addcontentsline{toc}{section}{Cписок использованных источников}

\sectioncentered*{Cписок использованных источников}
\addcontentsline{toc}{section}{Cписок использованных источников}


% \section{Cписок использованных источников}

[1] Уберизация производства [Электронный ресурс]. - Режим доступа: https://www.forbes.ru/brandvoice/sap/346251-uberizaciya-proizvodstva. (Дата обращения: 4.04.202)

[2] Анализ предметной области и требования к ПО [Электронный ресурс]. - Режим доступа: https://www.intuit.ru/studies/cou  \linebreak rses/64/64/lecture/1872. (Дата обращения: 04.04.2020)

[1] А.А.Волосевич. Архитектура программного обеспечения: Курс лекций для студентов специальности 1-40 01 03 Информатика и технологии программирования / А.А.Волосевич. — МИНСК : БГУИР,2013.

[3] М. Г. Лужецкий. ОС и Сети Прикладная информатика. Научные статьи /  М. Г. Лужецкий. - МОСКВА, 2006.

[4] Что такое электронная коммерция [Электронный ресурс]. - Режим доступа: http://kunegin.com/ref6/ecom/1.htm. (Дата обращения: 04.05.2020)

[5] Мировой рынок eCommerce в 2019 году вырос на 18\%: аналитика Internet Retailer [Электронный ресурс]. - Режим доступа: https://belretail.by \linebreak/news/mirovoy-ryinok-ecommerce-v-godu-vyiros-na-analitika-internet-retailer. \linebreak (Дата обращения: 04.05.2020)

[6] Global ecommerce sales to reach nearly \$3.46 trillion in 2019 [Электронный ресурс]. - Режим доступа: https://www.digitalcommerce360.com/artic \linebreak le/global-ecommerce-sales/. (Дата обращения: 04.05.2020)

[7] [Электронный ресурс]. - Режим доступа: . (Дата обращения: 04.05.2020)
Русская сеть. Информационный портал о программировании [Электронный ресурс] - URL: http://www.ruweb.ru/toclients/functions.shtml.

[8] R24.by [Электронный ресурс]. - Режим доступа: https://r24.by. (Дата обращения: 05.05.2020)

[9] William S. Vincent. Django for Professionals: Production websites with Python \& Django / William S. Vincent - 2019.
% https://www.amazon.com/Django-Professionals-Production-websites-Python-ebook/dp/B07VG4R5HR/ref=sr_1_1?dchild=1&keywords=django+2&qid=1590354726&s=digital-text&sr=1-1

[10] Nigel Poulton. Docker Deep Dive: Zero to Docker in a single book. Nigel Poulton / 2017.
% https://www.amazon.com/Docker-Deep-Dive-Nigel-Poulton-ebook/dp/B01LXWQUFF/ref=sr_1_1?dchild=1&keywords=docker+book&qid=1590354773&s=digital-text&sr=1-1


[11] Alex Banks. Learning React: Functional Web Development with React and Redux / Alex Banks - 2017.
% https://www.amazon.com/Learning-React-Functional-Development-Redux-ebook/dp/B071HB1526/ref=sr_1_2?dchild=1&keywords=react+book&qid=1590354819&s=digital-text&sr=1-2

[12] Firebase [Электронный ресурс]. - Режим доступа: https://fireba  \linebreak se.google.com/docs/reference/js. (Дата обращения: 08.05.2020)

[13] Monolithic vs Microservices architecture [Электронный ресурс]. - Режим доступа: https://www.geeksforgeeks.org/monolithic-vs-microservices-\\architecture/. (Дата обращения: 10.05.2020)

[14] Дейт К. Дж. Введение в системы баз данных / Дейт К. Дж - 2005.
% Дейт К. Дж. Введение в системы баз данных = Introduction to Database Systems. — 8-е изд. — М.: Вильямс, 2005. — 1328 с. — ISBN 5-8459-0788-8 (рус.) 0-321-19784-4 (англ.).

[15] Почему не стоит использовать двухуровневую архитектуру при разработке клиент-серверных приложений [Электронный ресурс]. - Режим доступа: https://habr.com/ru/post/348946/. (Дата обращения: 10.05.2020)

[16] Введение в Enterprise-разработку [Электронный ресурс]. - Режим доступа: https://javarush.ru/groups/posts/2519-chastjh-2-pogovorim-nemnogo-ob-arkhitekture-po. (Дата обращения: 10.05.2020)

[17] Redux [Электронный ресурс]. - Режим доступа: https://redux.js.org/. (Дата обращения: 15.05.2020)

[18] Redux Saga [Электронный ресурс]. - Режим доступа: https://redux-saga.js.org/. (Дата обращения: 15.05.2020)

[19] Simple Nested API Using Django REST Framework [Электронный ресурс]. - Режим доступа: https://apptension.com/blog/2017/09/13/rest-api-using-django-rest-framework/. (Дата обращения: 20.05.2020)

[20] Волосевич, А.А. Шаблоны и архитектура программ / А.А. Волосе- вич. — 2010. — 84 с.

[21] Палицын, В.А. Технико-экономическое обоснование дипломных проектов: Метод. пособие для студ. всех спец. БГУИР. В 4-х ч. Ч. 4: Про- екты программного обеспечения / В.А. Палицын. — Минск : БГУИР, 2006. — 76 с.

[22] Горовой, В.Г. Экономическое обоснование проекта по разработке программного обеспечения / В.Г. Горовой, А.В. Грицай, В.А. Пархименко. — Минск : БГУИР, 2014. — 12 с.
