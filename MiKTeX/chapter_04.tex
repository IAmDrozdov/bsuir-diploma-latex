% === Начало вычислений ===

% Зачем: вычисление времени на разработку.
\FPeval{\devTimePerDay}{8}
\FPeval{\devTimeInMonths}{4}
\FPeval{\devTimeInHours}{clip(\devTimeInMonths * 40 * 4)}

% Зачем: назначение оплаты для программиста и ведущего программиста.
\FPeval{\rankOneSalaryPerHours}{11.5}
\FPeval{\rankTwoSalaryPerHours}{8.32}
\FPeval{\bonusRate}{1.3}

% Зачем: вычисление основной заработной платы.
\FPeval{\rankOneSalary}{clip(\rankOneSalaryPerHours * \devTimeInHours * \bonusRate)}
\FPeval{\rankTwoSalary}{clip(\rankTwoSalaryPerHours * \devTimeInHours * \bonusRate)}
\FPeval{\totalSalary}{clip(\rankOneSalary + \rankTwoSalary)}

% Зачем: вычисление дополнительной заработной платы в размере 10%.
\FPeval{\addedSalaryNormative}{10}
\FPeval{\addedSalaryExact}{\totalSalary * \addedSalaryNormative / 100}
\FPround{\addedSalary}{\addedSalaryExact}{2}

% Зачем: вычисление отчислений в фонд социальной защиты и страхование в размере 34.6%.
\FPeval{\socialCoverageNormative}{0.6}
\FPeval{\socialProtectionNormative}{34}
\FPeval{\socialNormative}{clip(\socialCoverageNormative + \socialProtectionNormative)}
\FPeval{\socialNormativeCostExact}{(\totalSalary + \addedSalary) * \socialNormative / 100}
\FPround{\socialNormativeCost}{\socialNormativeCostExact}{2}

% Зачем: вычисление стоимости машинного времени.
\FPeval{\oneHourMachineTimeCost}{0.15}
\FPeval{\machineTimeCost}{clip(\oneHourMachineTimeCost * \devTimeInHours * \devTimePerDay * 2)}

% Зачем: вычисление накладных расходов в размере 50%.
\FPeval{\overheadCostNormative}{50}
\FPeval{\overheadCostExact}{\totalSalary * \overheadCostNormative / 100}
\FPround{\overheadCost}{\overheadCostExact}{2}

% Зачем: вычисление общую сумму расходов на создание программного модуля.
\FPeval{\overallCost}{clip(\totalSalary + \addedSalary + \socialNormativeCost + \machineTimeCost + \overheadCost)}

% Зачем: вычисление расходов на сопровождение программного модуля в размере 20%.
\FPeval{\supportCostNormative}{20}
\FPeval{\softwareSupportCostExact}{\totalSalary * \supportCostNormative / 100}
\FPround{\softwareSupportCost}{\softwareSupportCostExact}{2}

% Зачем: вычисление полной себестоимость программного модуля.
\FPeval{\baseCost}{clip(\overallCost + \softwareSupportCost)}

% Зачем: вычисление экономия затрат на заработную плату.
\FPeval{\patientsAdmittedCount}{20000}

\FPeval{\salaryAddedNormative}{10}
\FPeval{\salaryAddedCalcedExact}{(1 + (\salaryAddedNormative) / 100)}
\FPround{\salaryAddedCalced}{\salaryAddedCalcedExact}{2}

\FPeval{\salaryDeductedNormative}{34.6}
\FPeval{\salaryDeductedCalcedExact}{(1 + (\salaryDeductedNormative) / 100)}
\FPround{\salaryDeductedCalced}{\salaryDeductedCalcedExact}{2}

\FPeval{\laborAfter}{0.2}
\FPeval{\laborBefore}{0.5}
\FPeval{\laborDiff}{clip(\laborBefore - \laborAfter)}

\FPeval{\costSavingsExact}{\bonusRate * \laborDiff * \rankOneSalaryPerHours * \patientsAdmittedCount * \salaryAddedCalced * \salaryDeductedCalced}
\FPround{\costSavings}{\costSavingsExact}{2}

% Зачем: вычисление прироста чистой прибыли.
\FPeval{\profitIncExact}{\costSavings * (1 - 3 / 12)}
\FPround{\profitInc}{\profitIncExact}{2}

\FPeval{\incomeTaxRate}{18}
\FPeval{\profitIncFullYearExact}{\costSavings - \costSavings * \incomeTaxRate / 100}
\FPround{\profitIncFullYear}{\profitIncFullYearExact}{2}

% Зачем: вычисление показателей экономической эффективности.
\FPeval{\economicsEffectIndexPercent}{13}
\FPeval{\economicsEffectIndex}{0.13}

\FPeval{\economicsEffectIndexOneExact}{clip((1 + \economicsEffectIndex) ^ (1 - 1))}
\FPround{\economicsEffectIndexOne}{\economicsEffectIndexOneExact}{2}

\FPeval{\economicsEffectIndexTwoExact}{clip((1 + \economicsEffectIndex) ^ (1 - 2))}
\FPround{\economicsEffectIndexTwo}{\economicsEffectIndexTwoExact}{2}

\FPeval{\economicsEffectIndexThreeExact}{clip((1 + \economicsEffectIndex) ^ (1 - 3))}
\FPround{\economicsEffectIndexThree}{\economicsEffectIndexThreeExact}{2}

\FPeval{\profitIncWithOneIndex}{clip(\profitInc * \economicsEffectIndexOneExact)}
\FPeval{\profitIncFullYearWithTwoIndex}{clip(\profitIncFullYear * \economicsEffectIndexTwo)}
\FPeval{\profitIncFullYearWithThreeIndex}{clip(\profitIncFullYear * \economicsEffectIndexThree)}

\FPeval{\chddIndexOne}{clip(\profitIncWithOneIndex - \baseCost)}
\FPeval{\chddIndexTwo}{clip(\chddIndexOne + \profitIncFullYearWithTwoIndex)}
\FPeval{\chddIndexThree}{clip(\chddIndexTwo + \profitIncFullYearWithThreeIndex)}

% Зачем: вычисление рентабельности инвестиций в разработку.
\FPeval{\paybackInYears}{3}
\FPeval{\netProfitExact}{(\profitIncWithOneIndex + \profitIncFullYearWithTwoIndex + \profitIncFullYearWithThreeIndex) / \paybackInYears}
\FPround{\netProfit}{\netProfitExact}{2}

\FPeval{\roiExact}{(\netProfit / \baseCost) * 100}
\FPround{\roi}{\roiExact}{2}

% === Конец вычислений ===

\section{ТЕХНИКО\hyphЭКОНОМИЧЕСКОЕ ОБОСНОВАНИЕ}
\label{sec:economics}

\subsection{Краткая характеристика программного продукта}
\label{sub:economics:description}

Программное обеспечение, разрабатываемое в рамках дипломного проекта для автоматизации процесса сдачи в аренду товаров позволяет безопасно сдавать собственные вещи в аренду физическим лицам, контролируя процесс сдачи от размещения товара и сдачи его в аренду и до приемки товара обратно.
В качестве аналога можно рассматривать сервисы, для сдачи вещей конкретного назначения в аренду(холодильники, автомобили и тп), однако на данном сервисе возможно размещение неограниченного вида товаров.
Также за счет модульности данного программного обеспечения, а именно предоставления базовых операций, над товарами и модуля аутентификации пользователей данное приложение можно использовать как подсистему в программный продуктах.

Программное обеспечение обеспечивает выполнение следующих функций:
\begin{itemize}
  \item Регистрация пользователя в приложении;
  \item Размещение товаров для сдачи в аренду;
  \item Покупка товаров в аренду;
  \item Контроль за перемещаемыми товарами со стороны администратора;
  \item Отслеживание статуса сданного товара в аренду;
\end{itemize}

Разработка и внедрение данного программного обеспечения позволит ускорить поиск товара, за счет унификации трафика товаров, уменьшить количество нелегальных сделок.

Разработанное программное обеспечение представляет из себя два модуля: веб-приложение, написанное на React, являющееся графическим интерфейсом для доступа к программному интерфейсу, написанному на Python/Django, который взаимодействует с базой данных Postgres.

Программное обеспечение обеспечит прирост прибыли за счет унификации трафика сдачи вещей в аренду, и, как следствие увеличения количества сделок.
Также данное программное обеспечение можно использовать как основу, для создания платформы для контроля сдачи в аренду конкретного типа товаров, урезав спектр продуктов, доступных для оборота.

\subsection{Расчет затрат на разработку продукта}
\label{sub:economics:dev_cost}

Для разработки данного программного модуля необходимо привлечь два специалиста: программист, разрабатывающий серверную часть и программист, разрабатывающий сайт.
Длительность разработки составляет три месяца или $ \num{\devTimeInHours} $ часов рабочего времени.

Основная заработная плата исполнителей проекта на конкретное программное обеспечение определяется по следующей формуле:
\begin{equation}
  \label{eq:sub:economics:dev_cost:total_salary}
  \text{З}_{\text{о}} =
    \sum^{n}_{i = 1}
      \text{Т}_{\text{чi}} \cdot
      \text{Т}_{\text{ч}} \cdot
      \text{Ф}_{\text{эi}} \cdot
      \text{К}_{i} \text{\,,}
\end{equation}
\begin{explanation}
  где & $ \text{n} $ & количество исполнителей, занятых разработкой продукта; \\
      & $ \text{Т}_{\text{чi}} $ & часовая тарифная ставка $ i $-го исполнителя,~р.; \\
      & $ \text{Т}_{\text{ч}} $ & количество часов работы в день,~ч.; \\
      & $ \text{Ф}_{\text{эi}} $ & эффективный фонд рабочего времени $ i $-го исполнителя,~дн.; \\
      & $ \text{К}_{i} $ & коэффициент премирования.
\end{explanation}

В настоящий момент часовая тарифная ставка программиста, разрабатывающего серверную часть составляет $ \SI{\rankOneSalaryPerHours}{\text{р.}} $ и программиста, разрабатывающего сайт, действующая в компании, составляет $ \SI{\rankTwoSalaryPerHours}{\text{р.}} $.
Учитывая длительность разработки проекта и месячные тарифные ставки исполнителей, можем рассчитать основную заработную плату.
Подставив известные значения в формулу~(\ref{eq:sub:economics:dev_cost:total_salary}) и приняв коэффициент премирования $ \text{К}_{i} = \num{\bonusRate} $, вычислим основную заработную плату.
Данный расчет представлен в таблице~\ref{tbl:sub:economics:dev_cost:total_salary_calc}.

\begin{table}[ht]
  \caption{Расчет основной заработной платы исполнителей проекта.}
  \label{tbl:sub:economics:dev_cost:total_salary_calc}
  \begin{tabular}{| >{\centering}m{0.19\textwidth}
                  | >{\centering}m{0.16\textwidth}
                  | >{\centering}m{0.20\textwidth}
                  | >{\centering}m{0.18\textwidth}
                  | >{\centering\arraybackslash}m{0.15\textwidth}|}
    \hline
    Исполнитель
      & Часовая тарифная ставка,~р.
      & Плановый фонд рабочего времени,~ч.
      & Коэффициент премирования
      & Основная заработная плата,~р. \\
    \hline
    Программист, разрабатывающий сайт
      & $ \num{\rankTwoSalaryPerHours} $
      & $ \num{\devTimeInHours} $
      & $ \num{\bonusRate} $
      & $ \num{\rankTwoSalary} $\\
    \hline
    Программист, разрабатывающий серверную часть
      & $ \num{\rankOneSalaryPerHours} $
      & $ \num{\devTimeInHours} $
      & $ \num{\bonusRate} $
      & $ \num{\rankOneSalary} $ \\
    \hline
    Итого & - & - & - & $ \num{\totalSalary} $ \\
    \hline
  \end{tabular}
\end{table}

Дополнительная заработная плата исполнителей проекта, задействованных в разработке данного продукта, включает выплаты, предусмотренные законодательством о труде.
Она определяется по нормативу в процентах от основной заработной платы и вычисляется по следующей формуле:
\begin{equation}
  \label{eq:sub:economics:dev_cost:added_salary}
  \text{З}_{\text{д}} =
    \frac{\text{З}_{\text{о}} \cdot \text{Н}_{\text{д}}}
      {\num{100\%}} \text{\,,}
\end{equation}
\begin{explanation}
  где & $ \text{З}_{\text{о}} $ & основная заработная плата исполнителей проекта,~р.; \\
      & $ \text{Н}_{\text{д}} $ & норматив дополнительной заработной платы,~$ \% $.
\end{explanation}

Приняв норматив дополнительной заработной платы $ \text{Н}_{\text{д}} = \num{\addedSalaryNormative\%} $ и подставив известные данные в формулу~(\ref{eq:sub:economics:dev_cost:added_salary}), получим:
\begin{equation*}
  \text{З}_{\text{д}} =
    \frac{\num{\totalSalary} \times \num{\addedSalaryNormative\%}}
      {\num{100\%}} =
  \SI{\addedSalary}{\text{р.}}
\end{equation*}

Согласно действующему законодательству, отчисления в фонд социальной защиты населения составляют $ \num{\socialProtectionNormative\%} $, в фонд обязательного страхования "--- $ \num{\socialCoverageNormative\%} $ от фонда основной и дополнительной заработной платы исполнителей.
Общие отчисления на социальную защиту рассчитываются по следующей формуле:
\begin{equation}
  \label{eq:sub:economics:dev_cost:social_normative}
  \text{З}_{\text{сз}} =
    \frac{(\text{З}_{\text{о}} + \text{З}_{\text{д}}) \cdot \text{Н}_{\text{сз}}}
      {\num{100\%}} \text{\,,}
\end{equation}
\begin{explanation}
  где & $ \text{З}_{\text{о}} $ & основная заработная плата исполнителей проекта,~р.; \\
      & $ \text{З}_{\text{д}} $ & дополнительная заработная плата исполнителей проекта,~р.; \\
      & $ \text{Н}_{\text{сз}} $ & норматив отчисления на социальную защиту.
\end{explanation}

Приняв норматив отчисления на социальную защиту $ \text{Н}_{\text{сз}} = \num{\socialNormative\%} $ и подставив известные данные в формулу~(\ref{eq:sub:economics:dev_cost:social_normative}), получим:
\begin{equation*}
  \text{З}_{\text{сз}} =
    \frac{(\num{\totalSalary} + \num{\addedSalary}) \times \num{\socialNormative\%} }
      {\num{100\%}} =
  \SI{\socialNormativeCost}{\text{р.}}
\end{equation*}

Расходы по статье <<машинное время>> включают оплату машинного времени, необходимого для разработки и отладки данного программного модуля, и определяются по следующей формуле:
\begin{equation}
  \label{eq:sub:economics:dev_cost:machine_time}
  \text{Р}_{\text{м}} =
    \text{Ц}_{\text{м}} \cdot \text{Т}_{\text{ч}} \cdot \text{С}_{\text{р}} \text{\,,}
\end{equation}
\begin{explanation}
  где & $ \text{Ц}_{\text{м}} $ & цена одного часа машинного времени,~р.; \\
      & $ \text{Т}_{\text{ч}} $ & количество часов работы в день,~ч.; \\
      & $ \text{С}_{\text{р}} $ & длительность проекта,~дн.
\end{explanation}

Цена одного часа машинного времени составляет $ \text{Ц}_{\text{м}} = \SI{\oneHourMachineTimeCost}{\text{рублей}} $.
Разработка проекта займет $ \num{\devTimeInHours} $ часов.
Согласно действующему законодательству, количество рабочего времени в день составляет $ \text{Т}_{\text{ч}} = \SI{\devTimePerDay}{\text{часов}} $.
Подставляя известные данные в формулу~(\ref{eq:sub:economics:dev_cost:machine_time}), получаем:
\begin{equation*}
  \text{Р}_{\text{м}} =
    \num{\oneHourMachineTimeCost} \cdot \num{\devTimePerDay} \cdot 2 \cdot \num{\devTimeInHours} =
  \SI{\machineTimeCost}{\text{р.}}
\end{equation*}

Статья <<накладные расходы>> учитывает расходы, необходимые для содержания аппарата управления, вспомогательных хозяйств и опытных производств, а также расходы на общехозяйственные нужды.
Данная статья затрат рассчитывается по нормативу от основной заработной платы и вычисляется по формуле:
\begin{equation}
  \label{eq:sub:economics:dev_cost:overhead_cost}
  \text{Р}_{\text{н}} =
    \frac{\text{З}_{\text{о}} \cdot \text{Н}_{\text{рн}}}
      {\num{100\%}} \text{\,,}
\end{equation}
\begin{explanation}
  где & $ \text{З}_{\text{о}} $ & основная заработная плата исполнителей проекта,~р.; \\
      & $ \text{Н}_{\text{рн}} $ & норматив накладных расходов в организации,~$ \% $.
\end{explanation}

Приняв норму накладных расходов $ \text{Н}_{\text{рн}} = \num{\overheadCostNormative\%} $ и подставив известные данные в формулу~(\ref{eq:sub:economics:dev_cost:overhead_cost}), получаем:
\begin{equation*}
  \text{Р}_{\text{н}} =
    \frac{\num{\totalSalary} \times \num{\overheadCostNormative\%} }
      {\num{100\%}} =
  \SI{\overheadCost}{\text{р.}}
\end{equation*}

Общая сумма расходов по смете на создание программного модуля рассчитывается по формуле:
\begin{equation}
  \label{eq:sub:economics:dev_cost:overall_cost}
  \text{С}_{\text{р}} =
    \text{З}_{\text{о}} +
    \text{З}_{\text{д}} +
    \text{З}_{\text{сз}} +
    \text{Р}_{\text{м}} +
    \text{Р}_{\text{н}} \text{\,.}
\end{equation}

Подставляя ранее вычисленные значения в формулу~(\ref{eq:sub:economics:dev_cost:overall_cost}), получаем:
\begin{equation*}
  \text{С}_{\text{р}} =
    \num{\totalSalary} + \num{\addedSalary} + \num{\socialNormativeCost} + \num{\machineTimeCost} + \num{\overheadCost} =
  \SI{\overallCost}{\text{р.}}
\end{equation*}

Кроме того, организация\hyphразработчик осуществляет затраты на сопровождение и адаптацию программного модуля, которые определяются по установленному нормативу и вычисляются по следующей формуле:
\begin{equation}
  \label{eq:sub:economics:dev_cost:software_support}
  \text{Р}_{\text{са}} =
    \frac{\text{С}_{\text{р}} \cdot \text{Н}_{\text{рса}}}
      {\num{100\%}} \text{\,,}
\end{equation}
\begin{explanation}
  где & $ \text{С}_{\text{р}} $ & общая сумма расходов на разработку продукта,~р.; \\
      & $ \text{Н}_{\text{рса}} $ & норматив расходов на сопровождение и адаптацию проекта,~$ \% $.
\end{explanation}

Приняв значение норматива расходов на сопровождение и адаптацию программного обеспечения $ \text{Н}_{\text{рса}} = \num{\supportCostNormative\%} $ и подставив ранее вычисленные значения в формулу~(\ref{eq:sub:economics:dev_cost:software_support}), получаем:
\begin{equation*}
  \text{Р}_{\text{са}} =
    \frac{\num{\overallCost} \times \num{\supportCostNormative\%}}
      {\num{100\%}} =
  \SI{\softwareSupportCost}{\text{р.}}
\end{equation*}

Полная себестоимость создания программного обеспечения включает сумму затрат на разработку, сопровождение и адаптацию, и вычисляется по следующей формуле:
\begin{equation}
  \label{eq:sub:economics:dev_cost:base_cost}
  \text{С}_{\text{п}} = \text{С}_{\text{р}} + \text{Р}_{\text{са}} \text{\,.}
\end{equation}

Подставляя известные значения в формулу~(\ref{eq:sub:economics:dev_cost:base_cost}), получаем:
\begin{equation*}
  \text{С}_{\text{п}} =
    \num{\overallCost} + \num{\softwareSupportCost} =
  \SI{\baseCost}{\text{р.}}
\end{equation*}

Так как разработка программного модуля будет использоваться внутри компании, то затраты на разработку составят $ \num{\baseCost} $ рублей.

\subsection{Расчет стоимостной оценки результата}
\label{sub:economics:result_eval}

Результатом в сфере использования программного продукта является прирост чистой прибыли.
Прирост прибыли обеспечивается за счет увеличения количества сделок.
Экономия затрат использования программного модуля в расчете на объем выполняемых работ определяется по формуле:
\begin{equation}
  \label{eq:sub:economics:result_eval:cost_savings}
  \text{Э}_{\text{з}} =
    \text{К}_{\text{пр}} \cdot (\text{t}_{\text{c}} - \text{t}_{\text{н}}) \cdot \text{Т}_{\text{ч}} \cdot \text{N}_{\text{п}} \cdot
      (1 + \frac{\text{Н}_{\text{д}}}{\num{100\%}}) \cdot (1 + \frac{\text{Н}_{\text{но}}}{\num{100\%}}) \text{\,,}
\end{equation}
\begin{explanation}
  где & $ \text{К}_{\text{пр}} $ & коэффициент премий; \\
      & $ \text{t}_{\text{c}}, \text{t}_{\text{н}} $ & трудоемкость выполнения работы до и после внедрения программного продукта,~нормо\hyphчас; \\
      & $ \text{Т}_{\text{ч}} $ & часовая тарифная ставка, соответствующая разряду выполняемых работ,~р./ч.; \\
      & $ \text{N}_{\text{п}} $ & плановый объем работ,~шт.; \\
      & $ \text{Н}_{\text{д}} $ & норматив дополнительной заработной платы,~\%; \\
      & $ \text{Н}_{\text{но}} $ & норматив отчислений от заработной платы, включаемых в себестоимость,~\%.
\end{explanation}

До внедрения программно модуля, трудоемкость поиска и оформления сдачи товара в аренду составляла $ \num{\laborBefore} $ нормо \hyphчас, после внедрения "--- $ \num{\laborAfter} $ нормо\hyphчас.
В месяц сдается порядка $ \num{\patientsAdmittedCount} $ товаров.

Согласно действующему законодательству, норматив дополнительной заработной платы составляет $ \num{\salaryAddedNormative\%} $, норматив отчислений от заработной платы, включаемых в себестоимость "--- $ \num{\salaryDeductedNormative\%} $.
Учитывая данные нормативы, а также приняв коэффициент премий $ \text{К}_{\text{пр}} = \num{\bonusRate} $ и подставив ранее известные значения в формулу~(\ref{eq:sub:economics:result_eval:cost_savings}), вычислим прирост прибыли за счет использования данного программного продукта:
\begin{equation*}
  \text{Э}_{\text{з}} =
    \num{\bonusRate} \cdot (\num{\laborBefore} - \num{\laborAfter}) \cdot \num{\rankOneSalaryPerHours} \cdot \num{\patientsAdmittedCount} \cdot \num{\salaryAddedCalced} \cdot \num{\salaryDeductedCalced} =
  \SI{\costSavings}{\text{р.}}
\end{equation*}

Прирост чистой прибыли, полученной в результате повышения количества сделок, определяется по следующей формуле:
\begin{equation}
  \label{eq:sub:economics:result_eval:profit_increase}
  \Delta\text{П}_{\text{ч}} =
    \text{П}_{\text{п}} - (\frac{\text{П}_{\text{п}} \cdot \text{Н}_{\text{п}}}{\num{100\%}}) \text{\,,}
\end{equation}
\begin{explanation}
  где & $ \text{П}_{\text{п}} $ & прирост прибыли за счет использования продукта,~р.; \\
      & $ \text{Н}_{\text{п}} $ & ставка налога на прибыль,~$ \% $.
\end{explanation}

Разработка данного программного модуля начата в 2020 году, внедрение планируется во втором квартале 2020 года, поэтому программное приложение может получить уже в 2020 году 75\% прибыли:
\begin{equation*}
  \text{П}_{\text{ч1}} =
    \num{\costSavings} \cdot (1 - \frac{\num{3}}{\num{12}}) =
  \SI{\profitInc}{\text{р.}}
\end{equation*}

Приняв ставку налога на прибыль $ \text{Н}_{\text{п}} = \num{\incomeTaxRate\%} $ и подставив ранее вычисленные значения в формулу~(\ref{eq:sub:economics:result_eval:profit_increase}), получаем прирост чистой прибыли за счет экономии материальных ресурсов составит в последующие годы:
\begin{equation*}
  \Delta\text{П}_{\text{ч}} =
    \num{\costSavings} - \frac{\num{\costSavings} \times \num{\incomeTaxRate\%}}{\num{100\%}} =
  \SI{\profitIncFullYear}{\text{р.}}
\end{equation*}

\subsection{Расчет показателей эффективности использования продукта}
\label{sub:economics:performance}

Для расчета показателей экономической эффективности использования программного модуля необходимо полученные суммы результата (прироста чистой прибыли) и затрат (капитальных вложений) по годам привести к единому времени "--- расчетному году (за расчетный год принят 2020 год) путем умножения результатов и затрат за каждый год на коэффициент приведения, который рассчитывается по формуле:
\begin{equation}
  \label{eq:sub:economics:performance:performance_indicator}
  \mathit{ALFA}_{\text{t}} =
    (1 + \text{E}_{\text{н}})^{\text{t}_{\text{p}} - \text{t}}
\end{equation}
\begin{explanation}
  где & $ \text{E}_{\text{н}} $ & норматив приведения разновременных затрат и результатов; \\
      & $ \text{t}_{\text{p}} $ & расчетный год, $ (\text{t}_{\text{p}} = 1) $; \\
      & $ \text{t} $ & номер года, результаты и затраты которого приводятся к расчетному $ (2020 =  1, 2021 = 2, 2022 = 3) $.
\end{explanation}

Приняв норматив приведения разновременных затрат и результатов $ \text{E}_{\text{н}} = \num{\economicsEffectIndexPercent\%} $ и подставив известные данные в формулу~(\ref{eq:sub:economics:performance:performance_indicator}), получим:
\begin{equation*}
  \begin{gathered}
    \mathit{ALFA}_{\text{1}} = (1 + \num{\economicsEffectIndex})^{1 - 1} = \num{\economicsEffectIndexOne} \text{\,,} \\
    \mathit{ALFA}_{\text{2}} = (1 + \num{\economicsEffectIndex})^{1 - 2} = \num{\economicsEffectIndexTwo} \text{\,,} \\
    \mathit{ALFA}_{\text{3}} = (1 + \num{\economicsEffectIndex})^{1 - 3} = \num{\economicsEffectIndexThree} \text{\,.}
  \end{gathered}
\end{equation*}

Результаты расчета показателей эффективности использования программного модуля приведены в таблице~\ref{tbl:sub:economics:performance:economic_effect_calc}.
\begin{table}[ht]
  \caption{Расчет экономического эффекта от использования продукта.}
  \label{tbl:sub:economics:performance:economic_effect_calc}
  \begin{tabular}{| >{\centering}m{0.40\textwidth}
                  | >{\centering}m{0.05\textwidth}
                  | >{\centering}m{0.16\textwidth}
                  | >{\centering}m{0.16\textwidth}
                  | >{\centering\arraybackslash}m{0.16\textwidth}|}
    \hline
    Показатели
      & Ед. изм.
      & 2020
      & 2021
      & 2022 \\
    \hline
    \raggedright{\textbf{Результат}} & & & & \\
    \hline
    \raggedright{1. Прирост чистой прибыли}
      & р.
      & $ \num{\profitInc} $
      & $ \num{\profitIncFullYear} $
      & $ \num{\profitIncFullYear} $ \\
    \hline
    \raggedright{2. Прирост результата}
      & р.
      & $ \num{\profitInc} $
      & $ \num{\profitIncFullYear} $
      & $ \num{\profitIncFullYear} $ \\
    \hline
    \raggedright{3. Коэффициент приведения}
      & ед.
      & $ \num{\economicsEffectIndexOne} $
      & $ \num{\economicsEffectIndexTwo} $
      & $ \num{\economicsEffectIndexThree} $ \\
    \hline
    \raggedright{4. То же с учетом времени}
      & р.
      & $ \num{\profitIncWithOneIndex} $
      & $ \num{\profitIncFullYearWithTwoIndex} $
      & $ \num{\profitIncFullYearWithThreeIndex} $ \\
    \hline

    \raggedright{\textbf{Затраты (инвестиции)}} & & & & \\
    \hline
    \raggedright{5. Инвестиции в разработку}
      & р.
      & $ \num{\baseCost} $
      & -
      & - \\
    \hline
    \raggedright{5. Всего инвестиций}
      & р.
      & $ \num{\baseCost} $
      & -
      & - \\
    \hline
    \raggedright{6. То же с учетом времени}
      & р.
      & $ \num{\baseCost} $
      & -
      & - \\
    \hline
    \raggedright{\textbf{Экономический эффект}} & & & & \\
    \hline
    \raggedright{7. Чистый приведенный доход}
      & р.
      & $ \num{\chddIndexOne} $
      & $ \num{\profitIncFullYearWithTwoIndex} $
      & $ \num{\profitIncFullYearWithThreeIndex} $ \\
    \hline
    \raggedright{8. То же нарастающим итогом}
      & р.
      & $ \num{\chddIndexOne} $
      & $ \num{\chddIndexTwo} $
      & $ \num{\chddIndexThree} $ \\
    \hline
  \end{tabular}
\end{table}

Зная инвестиции в разработку программного модуля, разрабатываемого в данном дипломном проекте, можно расчитать рентабельность инвестиций.
Рентабельности инвестиций в разработку и внедрение данного программно продукта можно расчитать по следующей формуле:
\begin{equation}
  \label{eq:sub:economics:result_eval:roi}
  \text{Р}_{\text{и}} =
    \frac{\text{П}_{\text{чср}}}{\text{С}_{\text{п}}} \times \num{100\%} \text{\,,}
\end{equation}
\begin{explanation}
  где & $ \text{С}_{\text{п}} $ & полная себестоимость создания продукта,~р; \\
      & $ \text{П}_{\text{чср}} $ & среднегодовая величина чистой прибыли за расчетный период времени,~р.
\end{explanation}

Среднегодовая величина чистой прибыли за расчетный период времени использования программного модуля рассчитывается по следующей формуле:
\begin{equation}
  \label{eq:sub:economics:result_eval:net_profit}
  \text{П}_{\text{чср}} =
    \frac{\sum^{n}_{i = 1} \text{П}_{\text{чi}}}{\text{n}} \text{\,,}
\end{equation}
\begin{explanation}
  где & $ \text{n} $ & количество лет, необходимое для окупаемости продукта,~шт.; \\
      & $ \text{П}_{\text{чi}} $ & чистая прибыль, полученная в $ i $-го году,~р.
\end{explanation}

Приняв количество лет, необходимое для окупаемости продукта $ \text{n} = \num{\paybackInYears} $ года и подставив ранее вычисленные значения в формулу~(\ref{eq:sub:economics:result_eval:net_profit}), получаем:
\begin{equation*}
  \text{П}_{\text{чср}} =
    \frac{\num{\profitIncWithOneIndex} + \num{\profitIncFullYearWithTwoIndex} + \num{\profitIncFullYearWithThreeIndex}}{\num{\paybackInYears}} =
  \SI{\netProfit}{\text{р.}}
\end{equation*}

Таким образом, рентабельность инвестиций в разработку данного программного модуля будет равна:
\begin{equation*}
  \text{Р}_{\text{и}} =
    \frac{\num{\netProfit}}{\num{\baseCost}} \times \text{\num{100\%}} =
  \SI{\roi}{\text{\%}}
\end{equation*}

В результате технико\hyphэкономического обоснования применения программного модуля для автоматизации процесса сдачи в аренду товаров были получены следующие значения показателей эффективности:
\begin{itemize}
  \item Рентабельность инвестиций составляет $ \num{\roi}\% $.
  \item Затраты на разработку окупятся на третий год его использования.
\end{itemize}

Таким образом, разработка и применение программного модуля является эффективной.
Данные инвестиции в разработку и внедрение целесообразно осуществлять.
