\lstset{style=fsharpstyle}

\section{Используемые технологии}
Прежде чем приступать к новому проекту, следует оценить, какой язык или фреймворк лучше всего подойдет для достижения желаемого результата.
Что  наиболее важно: безопасность, скорость разработки, масштабируемость, универсальность, поддержка.
Лучше принять информированное решение перед тем как приступать к работе, чем потом раскаиваться в поспешном выводе о выборе технологий.

Существует несколько тезисов, которые обосновывают Django, как лучший фреймворк для написания веб-приложений:
\begin{itemize}
    \item Требуется быстро работать, быстро развертывать и вносить изменения в проект по ходу работы;
    \item По умолчанию приложение должно быть защищено от наиболее распространенных уязвимостей и атак, в частности: CSRF, SQL-инъекции, XSS, кликджекинг;
    \item В любой момент в приложении может потребоваться масштабирование: как наращивание, так и сокращение;
    \item В перспективе планируется интегрировать новейшие технологии, например, машинное обучение;
    \item Нужно использовать надежный фреймворк, который активно разрабатывается, используется многими топовыми компаниями и ведущими веб-сайтами во всем мире;
    \item нужна поддержка ORM.
\end{itemize}

В силу того, что ReactJs имеет большое сообщество разработчиков, обладает емким и понятным API, имеет большое количество готовых компонентов для разработки клиентского приложения был выбран React.

\subsection{Веб-фреймворк Django}
Django — свободный фреймворк для веб-приложений на языке Python, использующий шаблон проектирования MVC[9].

Сайт на Django строится из одного или нескольких приложений, которые рекомендуется делать отчуждаемыми и подключаемыми.
Это одно из существенных архитектурных отличий этого фреймворка от некоторых других (например, Ruby on Rails).
Один из основных принципов фреймворка — DRY.

Для работы с базой данных Django использует собственный ORM, в котором модель данных описывается классами Python, и по ней генерируется схема базы данных.

Архитектура Django похожа на «Модель-Представление-Контроллер» (MVC).
Контроллер классической модели MVC примерно соответствует уровню, который в Django называется Представление, а презентационная логика Представления реализуется в Django уровнем Шаблонов.
Из-за этого уровневую архитектуру Django часто называют «Модель-Шаблон-Представление» (MTV).
Первоначальная разработка Django как средства для работы новостных ресурсов достаточно сильно отразилась на его архитектуре: он предоставляет ряд средств, которые помогают в быстрой разработке веб-сайтов информационного характера.
Так, например, разработчику не требуется создавать контроллеры и страницы для административной части сайта, в Django есть встроенное приложение для управления содержимым, которое можно включить в любой сайт, сделанный на Django, и которое может управлять сразу несколькими сайтами на одном сервере.
Административное приложение позволяет создавать, изменять и удалять любые объекты наполнения сайта, протоколируя все совершённые действия, и предоставляет интерфейс для управления пользователями и группами (с пообъектным назначением прав).

В дистрибутив Django также включены приложения для системы комментариев, синдикации RSS и Atom, «статических страниц» (которыми можно управлять без необходимости писать контроллеры и представления), перенаправления URL и другое.

\subsection{Инструмент контейнеризации Docker}
Докер — это открытая платформа для разработки, доставки и эксплуатации приложений[10].
Docker разработан для более быстрого выкладывания ваших приложений.
С помощью docker вы можете отделить ваше приложение от вашей инфраструктуры и обращаться с инфраструктурой как управляемым приложением.
Docker помогает выкладывать ваш код быстрее, быстрее тестировать, быстрее выкладывать приложения и уменьшить время между написанием кода и запуска кода.
Docker делает это с помощью легковесной платформы контейнерной виртуализации, используя процессы и утилиты, которые помогают управлять и выкладывать ваши приложения.

В своем ядре docker позволяет запускать практически любое приложение, безопасно изолированное в контейнере.
Безопасная изоляция позволяет вам запускать на одном хосте много контейнеров одновременно.
Легковесная природа контейнера, который запускается без дополнительной нагрузки гипервизора, позволяет вам добиваться больше от вашего железа.

Платформа и средства контейнерной виртуализации могут быть полезны в следующих случаях: упаковывание вашего приложения (и так же используемых компонент) в docker контейнеры; раздача и доставка этих контейнеров вашим командам для разработки и тестирования; выкладывания этих контейнеров на ваши продакшены, как в дата центры так и в облака.

Docker прекрасно подходит для организации цикла разработки.
Docker позволяет разработчикам использовать локальные контейнеры с приложениями и сервисами. Что впоследствии позволяет интегрироваться с процессом постоянной интеграции и выкладывания (continuous integration and deployment workflow).

Основанная на контейнерах docker платформа позволят легко портировать вашу полезную нагрузку.
Docker контейнеры могут работать на вашей локальной машине, как реальной так и на виртуальной машине в дата центре, так и в облаке.
Портируемость и легковесная природа docker позволяет легко динамически управлять вашей нагрузкой.
Docker можно использовать, чтобы развернуть или погасить ваше приложение или сервисы. Скорость Docker позволяет делать это почти в режиме реального времени.

\subsection{React}
React — JavaScript-библиотека с открытым исходным кодом для разработки пользовательских интерфейсов[11].

React разрабатывается и поддерживается Facebook, Instagram и сообществом отдельных разработчиков и корпораций.

React может использоваться для разработки одностраничных и мобильных приложений. Его цель — предоставить высокую скорость, простоту и масштабируемость.
В качестве библиотеки для разработки пользовательских интерфейсов React часто используется с другими библиотеками, такими как Redux.

Свойства передаются от родительских компонентов к дочерним.
Компоненты получают свойства как множество неизменяемых значений, поэтому компонент не может напрямую изменять свойства, но может вызывать изменения через callback функции.
Такой механизм называют «свойства вниз, события наверх».
React использует виртуальный DOM. React создает кэш структуру в памяти, что позволяет вычислять разницу между предыдущим и текущим состояниями интерфейса для оптимального обновления DOM браузера.
Таким образом программист может работать со страницей, считая, что она обновляется вся, но библиотека самостоятельно решает, какие компоненты страницы необходимо обновить.

JavaScript XML — это расширение синтаксиса JavaScript, которое позволяет использовать похожий на HTML синтаксис для описания структуры интерфейса.
Как правило, компоненты написаны с использованием JSX, но также есть возможность использования обычного JavaScript.
JSX напоминает другой язык, созданный в компании Фейсбук для расширения PHP, XHP .

React используется не только для отрисовки HTML в браузере.
Например, Facebook имеет динамические графики, которые отрисовываются в теги <canvas>.
Netflix и PayPal используют изоморфные загрузки для отрисовки идентичного HTML на сервере и клиенте.

\subsection{Firebase}
Firebase - проект корпорации Google, который является поставщиком облачных услуг для создания приложений[12].
Firebase помогает быстро создавать качественные приложения, увеличивать аудиторию вовлеченных пользователей и повышать доходы.
Платформа содержит множество полезных функций для вашего приложения, в том числе серверный код для мобильных сервисов, статистику, а также инструменты для монетизации и расширения аудитории.

Пакет разработчика Firebase объединяет интуитивно понятные API, избавляя вас от необходимости управлять отдельными пакетами.
Платформа, использующая инфраструктуру Google, предоставляет необходимые возможности для каждого этапа разработки и роста.

Главные преимущества:
\begin{itemize}
    \item Скорость работы.
    В пакете разработчика Firebase собраны интуитивно понятные API, которые упрощают и ускоряют разработку качественных приложений.
    Также в распоряжении пользователя имеются все необходимые инструменты для расширения пользовательской базы и повышения доходов;
    \item Готовая инфраструктура.
    Пользователю не требуется создавать сложную инфраструктуру или работать с несколькими панелями управления;
    \item Статистика.
     В основе Firebase лежит бесплатный аналитический инструмент, разработанный специально для мобильных устройств.
     Google Analytics для Firebase позволяет получать данные о действиях пользователей приложения и сразу же принимать меры с помощью дополнительных функций;
    \item Кроссплатформенность.
    Firebase работает на любых платформах благодаря пакетам разработчика для Android, iOS, JavaScript и C++.
    Обращаться к Firebase можно так же, используя серверные библиотеки или REST API;
    \item Масштабируемость.
    Если приложение станет популярным и нагрузка на него возрастет, нет необходимости менять код сервера или привлекать дополнительные ресурсы ‒ Firebase берет эту работу на себя.
    Кроме того, большинство функций Firebase бесплатны и останутся такими независимо от масштаба проектов.
    Платных функций четыре. В них предусмотрен бесплатный пробный период и два тарифных плана;
    \item Бесплатная поддержка по электронной почте.
    Кроме того, команда Firebase и специалисты по разработке Google ответят на вопросы на ресурсах Stack Overflow и GitHub.
\end{itemize}
